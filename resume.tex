%%%%%%%%%%%%%%%%%%%%%%%%%%%%%%%%%%%%%%%%%
% Medium Length Professional CV
% LaTeX Template
% Version 2.0 (8/5/13)
%
% This template has been downloaded from:
% http://www.LaTeXTemplates.com
%
% Original author:
% Trey Hunner (http://www.treyhunner.com/)
%
% Important note:
% This template requires the resume.cls file to be in the same directory as the
% .tex file. The resume.cls file provides the resume style used for structuring the
% document.
%
%%%%%%%%%%%%%%%%%%%%%%%%%%%%%%%%%%%%%%%%%

%----------------------------------------------------------------------------------------
%	PACKAGES AND OTHER DOCUMENT CONFIGURATIONS
%----------------------------------------------------------------------------------------

\documentclass{resume} % Use the custom resume.cls style

\usepackage[left=0.5in,top=0.5in,right=0.5in,bottom=0.5in]{geometry} % Document margins

%\usepackage[scaled]{helvet}
%\renewcommand\familydefault{\sfdefault}
%\usepackage[T1]{fontenc}
\usepackage{charter}

\name{Alok Hota} % Your name
%\address{123 Broadway \\ City, State 12345} % Your address
%\address{123 Pleasant Lane \\ City, State 12345} % Your secondary addess (optional)
\address{(615)~$\cdot$~419~$\cdot$~7781 \\ alok@utk.edu} % Your phone number and email

\begin{document}

%----------------------------------------------------------------------------------------
%	EDUCATION SECTION
%----------------------------------------------------------------------------------------

\begin{rSection}{Education}

{{\bf PhD in Computer Science}, University of Tennessee, 3.92 GPA} \hfill {\em in progress} \\ 
%\textbf{PhD in Computer Science}, 3.92 GPA \\
Focus on Large Data Visualization--scalable software platforms to support interactive \\
visualization of complex spatio-temporal datasets.

{{\bf MS in Computer Science}, University of Tennessee} \hfill {\em 2014--2017}

{{\bf BE in Computer Engineering}, Vanderbilt University} \hfill {\em 2011--2013}

{{\bf BS in Computer Science}, University of Tennessee} \hfill {\em 2008--2011}

\end{rSection}

%----------------------------------------------------------------------------------------
%	WORK EXPERIENCE SECTION
%----------------------------------------------------------------------------------------

\begin{rSection}{Experience}

\begin{rSubsection}{Sandia National Laboratories}{Summer 2016}{Graduate Student Intern}{Albuquerque, NM}
    \item Development of a method for general-case vectorization in VTK-m for x86 CPUs
\end{rSubsection}

%------------------------------------------------

\begin{rSubsection}{Intel Parallel Computing Center at Joint Institute for Computational Science}{Fall 2014--present}{Graduate Research Assistant}{Knoxville, TN}
    \item Integration and support of OSPRay ray tracing rendering engine into the VisIt \\ visualization application
\end{rSubsection}

%------------------------------------------------

\begin{rSubsection}{Advanced Computing Center for Research and Education}{June 2013--July 2014}{Professional Programmer}{Nashville, TN}
\item Developed live depot activity monitor for storage depots
\item Expanded in-house message queue system for parallel job distribution
\end{rSubsection}

%------------------------------------------------

\begin{rSubsection}{Institute for Software Integrated Systems}{February 2012--May 2013}{Undergraduate Student Researcher}{Nashville, TN}
\item Worked under Dr. Gautam Biswas to develop a traffic simulation model
\end{rSubsection}

%------------------------------------------------

\begin{rSubsection}{National Oceanic and Atmospheric Administration}{Summer 2011}{Undergraduate Student Intern}{Silver Spring, MD}
\item Developed several NOAA EVL Images of the Day
\item Processed and visualized lightning strikes during a tornado event
\end{rSubsection}

%------------------------------------------------

\begin{rSubsection}{Fisk University}{2009--2011}{Undergraduate Student Researcher}{Nashville, TN}
    \item Worked under Dr. Lei Qian on NOAA-funded projects
    \item Developed a hurricane detection algorithm using target location
\end{rSubsection}

\end{rSection}

%----------------------------------------------------------------------------------------
%	TECHNICAL STRENGTHS SECTION
%----------------------------------------------------------------------------------------

\begin{rSection}{Skills}

\begin{tabular}{ @{} >{\bfseries}r @{\hspace{3ex}} l }
Fluent with & \textbf{Python}, C/C++, Java, Bash \\
Knowledgeable with & VisIt, ParaView, VTK, VTK-m, CUDA
\end{tabular}

\end{rSection}

%----------------------------------------------------------------------------------------
%	EXAMPLE SECTION
%----------------------------------------------------------------------------------------

\begin{rSection}{Scholarships and Awards}

University of Tennessee Athletic Department Graduate Student Award \hfill \textit{2014--present} \\
Dean's List, Vanderbilt University \hfill \textit{2011, 2012} \\
Summa cum Laude, Fisk University \hfill \textit{2011} \\
Provost's List, Fisk University \hfill \textit{2008--2011}

\end{rSection}

%----------------------------------------------------------------------------------------
%	EXAMPLE SECTION
%----------------------------------------------------------------------------------------

\begin{rSection}{Publications}
\textbf{Alok Hota}, Mohammad Raji, Tanner Hobson, Jian Huang, ``A Space-Efficient Method for Ensemble Analysis and Visualization'', Accepted to \textit{EGPGV}, 2017.

Mohammad Raji, \textbf{Alok Hota}, Robert Sisneros, Jian Huang, ``Photo-Guided Exploration of Volume Data Features'', Accepted to \textit{EGPGV}, 2017.

Mohammad Raji, \textbf{Alok Hota}, Jian Huang, ``Embeddable Volume Rendering'', Submitted to \textit{IEEE SciVis}, 2017.
\end{rSection}

%----------------------------------------------------------------------------------------

\end{document}
